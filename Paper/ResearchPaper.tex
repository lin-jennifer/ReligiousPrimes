\documentclass[letterpaper,man,natbib,noextraspace,12pt]{apa6}  %Leave this stuff alone

\usepackage[english]{babel}
\usepackage[utf8x]{inputenc}
\usepackage{amsmath}
\usepackage{graphicx}
%\usepackage{mathpazo}
\usepackage{natbib}
\usepackage{hyperref}

%Below is what you can edit
\title{\large{May God’s Grace Guide Me: Research on the Impact of Subliminal Religious Primes on Political Issue Attitudes and Voting Decisions}}
\shorttitle{Religious Primes and Public Opinion}
\author{Jennifer Lin and Steven M. Graham}
\affiliation{New College of Florida}

\abstract{The influence of religion is embedded deeply in the soil of American political culture. The majority of Americans go to church and affiliate with some type of faith throughout their lives. Recently, there is also a rise in the use of churches as polling locations. While the intention is not to mix the notions of church and state, this raises the question of potential influences of religion on political behavior. In this research project, I aim to understand if priming of religion, as exemplified by voting in a church, would lead to more conservative voting behaviors. Through a survey design with 296 participants recruited through Amazon Mechanical Turk, this research primes participants to think about religion and asks them to rate their agreement with a vignette that presents an argument on abortion. The results suggest that the argument was more impactful in swaying people’s decisions, and not the religious prime itself. A possible reason for this is the saliency of the prime. As the prime was not as salient as voting in a church, this may not create the effect that the study aimed to test. For further research, it is useful to make more salient primes as a way to test the interaction of religion and conservative political behavior.}

\begin{document}
\maketitle

Each election year, voters are assigned to polling locations that are located near their place of residence. These polling location rosters consists of a handful of schools, libraries, office buildings, and churches \citep{barreto_are_2009}. These public places are often selected by public officials for their ease in space reservation. In the selection of these polling locations, the cost of voting may be affected. As individuals use personal predispositions to decide whether they should vote and to decide who to vote for, the results reflect an aggregate of self-identities that play a major role in the casting of the ballot. Individuals may use personal identities such as party affiliation, race, class, and gender to make their choices. Additionally, the United States is a very religious nation, with over 85\% of its people claiming allegiance to a major religious denomination \citep{domke_god_2008, putnam_american_2010} Individuals can be motivated to use their religious affiliation to govern their votes just as religion governs their life. When the person and place interact, decisions can be influenced by either or both of these factors. Therefore, does this space, along with American religious culture, play an impact in influencing voter decision. In this study, I aim to address this question. I am interested in understanding the influence of subliminal primes of polling location as a potential source of influence for political behavior. The governing research question for this study is twofold. How do subliminal primes influence political behavior? Do subliminal primes to religiosity influence people to support a more conservative approach to the issue at hand? I will test these questions through a survey that stimulates participants to a religious prime and asks them about their opinions on abortion. From this study, I find statistical significance in the relationship between the interaction of the prime and argument conditions. However, the significance lies more with the influence of the argument than the prime. But before I discuss the methods and results in greater detail, I will begin with a review of the literature in the field. This review will discuss the influence that religion has in American politics, and what priming of religion can mean for individual decision making at the ballot box along with political participation broadly construed. With this discussion, it will make it more beneficial to understand the influence of the variables and the results as presented in this study.

\section{American Grace: The Interaction of Religion and American Politics}

From data collected in a 2010 study \citep{putnam_american_2010}, 85\% of Americans identify with a major religious denomination. Within the general population of the American electorate, there is a mix of Christians, Catholics, Hindus, Jews, Buddhists and Muslims, among others. While Americans may be divided by faith, they are mostly united by a common allegiance towards a higher power. For believers, religion is central to their identity and their way of life \citep{wald_religion_2011}. Politicians are aware of this, and take each opportunity to integrate subliminal religious cues into their speeches and platforms to connect with their audience \citep{albertson_religious_2011, domke_god_2008}. The use of religious cues in political campaigns speaks to the importance of religious identities and their influence in candidate evaluations. It begs the question on why Americans internally favor politicians who appeal to politics \citep{albertson_religious_2011}, when they clearly dislike overt religious cues \citep{mclaughlin_cueing_2014}. In this section, I will explore this question by explaining the role that religion plays in the hearts and minds of Americans. I will also analyze the implications that this plays on the overall impact of references to God on American politics.

In American Grace: How Religion Divides and Unites Us \citep{putnam_american_2010}, the authors discuss the influence that religion has on American politics broadly construed. Their studies revolve around data collected from their 2010 Faith Matters Survey and ethnographical data from select US churches to construct their arguments. Their exploration comes to one key conclusion: Americans are very religious people and their interactions in religious settings influence the way they think, act and feel in non-religious atmospheres. As their observations from congregations suggest, fellow parishioners and clergy members influence political behavior in direct and indirect ways. For one, people who attend church are more likely to know more people. Building off previous work \citep{putnam_bowling_2001}, people who socialize more with others are more likely to be knowledgeable in current events and be more involved in politics \citep{huckfeldt2001social}. As social capital increases, especially from a religious standpoint, it makes it easier for religion and politics to intertwine \citep{putnam_american_2010}. Members of the clergy have leverage over urging their followers to support candidates who endorse the teachings of the Word. As these ideas get bounced off the followers’ networks, they are more likely to support politicians who belong to parties that adhere to these principles \citep{calfano_god_2009}.
Depending on the teachings of the particular church, parishioners can use their clergy’s endorsements of policy to make their decisions \citep{brown_race_2016}. This opens the possibility to exclusive rhetoric as a means to close believers off to new ideas as they “go against the Word” \citep{djupe_religious_2013}. By solidifying us-versus-them attitudes in this light, religion divides more than it binds. Politicians are aware of this, and will be motivated to appeal to religion more to secure their core voters \citep{calfano_god_2009, domke_god_2008} In the next section, I will discuss more regarding the implications of these appeals in the hearts and minds of the voters. But in this section, it is worthwhile to note that the reasons that religion becomes so deeply rooted in politics has reasons vested in the social networks of individual churchgoers. As people lose their interests for other civic organizations \citep{putnam_bowling_2001}, religious groups continue to serve the spiritual needs of individuals \citep{putnam_american_2010}, one that cannot be filled by the rise of social media. Therefore, just as religion will remain core to the identity of Americans, it will be as impactful on the shaping of American politics. This feature of individual identity will join the ranks of race, class, gender, and geography in shaping individual political opinions and nationwide political sentiments.

\section{Devoted Hearts and Souls: How Religious Primes Shape Religious Minds
}

Our environment provides cues that influence our thoughts (Kahneman, 2011), feelings, and behaviors \citep{cialdini_pre-suasion:_2016}. Not surprisingly, our environment can serve to influence our propensity to participate in politics \citep{barreto_are_2009}. Many factors influence political decision-making \citep{gelman_red_2010} including race \citep{mclaughlin_conditioned_2016}, gender \citep{lawless_it_2005}, and socioeconomic status \citep{gilens2012affluence}. These factors are key to a person’s identity and decisions as they are often ascribed or hard to achieve. However, an identity less explored by political behavior research is the influence of religion on political decisions. As I discussed in the previous section, religion plays a major role in American politics and it shapes political campaigns as politicians vie to win the hearts and minds of their followers. In this section, I will connect the influence of personal religious beliefs, religious primes, and its influence of voters at the ballot box. 

The environment in which we cast a vote influences the vote we cast \citep{berger_contextual_2008}. As most polling locations utilize easy to reserve spaces as venues, they often land in the hospitality of public buildings \citep{barreto_are_2009}. In an exploration of the influence of location on decision making \citep{berger_contextual_2008}, researchers found that people who voted in schools were more likely to support education-boosting policies. In another related study \citep{rutchick_deus_2010}, researchers found that voting in churches influenced more conservative voting patterns as voters were primed to vote according to the teachings of their core beliefs. 

When making decisions on the ballot, voters factor in a variety of conditions that influence whether they will cast a ballot in the first place, and who they will vote for. Hight costs are often associated with non-voting \citep{haspel_location_2005}. When the polling location is hard to find \citep{barreto_are_2009} or far from one’s home \citep{haspel_location_2005}, voters will not be interested to rush to the polling booth. The need to factor in these costs and find solutions to barriers to voting, people will most likely be more cognitively tired and vote based on heuristics such as religion and race \citep{weber_courting_2012}. 

Past research supports the idea that when people have high costs to voting, they will vote based on heuristics. When individuals are subliminally primed with cues that lead them to align themselves with race and religion, they are more likely to support these candidates when the costs of voting are high \citep{kam_implicit_2007}. Additionally, as the media associates the Republican party with religious ideas and politicians \citep{calfano_god_2009, mclaughlin_cueing_2014}, frequent exposure allows individual processing to make these connections, and subliminal primes will facilitate such processing \citep{kahneman_thinking_2012}. When frequent exposure leads to increased favorability, especially on a topic that the individual favors, it opens the possibility that individuals will be influenced by this idea. To apply this line of logic to religion in politics, individuals are frequently exposed to religion from their own practices. \citep{putnam_american_2010}. They are also frequently exposed to religion from politicians on the campaign trail \citep{domke_god_2008}. As they vote in a religious institution \citep{rutchick_deus_2010}, it is very likely that these connections can arise to influence their vote in accordance to their faith. This logic leads in to the goals of the present study. While voting in a church may not be a subtle prime, it carries an impact to an individual’s decisions. Therefore, this study is interested to see if truly subtle environments and primes at the polling place can remind individuals of their religious affiliation and urge them to vote according to it. 

\section{Present Research and Hypothesis
}

Throughout this paper, I’ve introduced the role of religion in American politics and the way religious identities shape political decisions. In the discussion, we see that individuals who see religion as a key component to their identity are more likely to participate in politics with fellow believers \citep{putnam_american_2010}. We also see that this form of civic participation yields similar forms of participation in the ballot box. This behavior is more amplified when we vote in places that subliminally prime support for specific issues \citep{berger_contextual_2008, rutchick_deus_2010}. This knowledge leads to the main research question and hypothesis. As I mentioned in the opening of this paper, the purpose of this study is to build on previous work on location primes and voting behavior \citep{berger_contextual_2008}. The main research question that this study is interested is: Does subliminal primes to religion influence political decision-making? And if it does, which way, on a left-right scale, does religion influence people to lean in their votes?

This question and past research informs my hypothesis. As we know from past research, the American people are quite religious \citep{putnam_american_2010} and such religiosity informs their political decisions \citep{rutchick_deus_2010}. We also know that subliminal primes can influence our thoughts \citep{kahneman_thinking_2012} and decisions \citep{albertson_religious_2011, kam_implicit_2007}. As people use religious cues to shape their decisions, it leads me to wonder about the implications such primes have towards the resulting vote. Therefore, to answer my research question, I hypothesize that (1) religious priming influences political decisions, and (2) people who are subliminally primed by religion will tend to vote more conservatively. This study utilizes a between -subjects survey design to test this question and hypotheses, which I will discuss in the next section. 

\section{Method
}

\subsection{Participants
}

A total of 356 participants were recruited to complete a survey through Amazon Mechanical Turk. In the results analysis, 60 of them were sorted out because they did not complete the questions as intended. At the end, a total of 296 responses were incorporated into the results analysis. In this sample. there was a total of 128 females. The average age for all participants is 36 years. Eighty percent of the participants self-identified as White. Most participants (n = 262) had some form of college education or graduate training. Forty-five percent of participants identified as Democrat. A minority of the participants (n = 90) identified as religious. In order to participate in the research, participants must be registered to work with Amazon Mechanical Turk. This is because each participant was rewarded \$1 for their participation that was paid to them through the program. Participants each gave their informed consent for their participation and understood that they were able to withdraw from the study at any time should they be uncomfortable in answering the questions. Per request of the Institutional Review Board, participants were given a content warning on abortion and mentions of religion as necessary to aid them in their decision to participate in the present study.

\subsection{Procedure}

Participants were invited to participate in a survey entitled “Public Opinion on Abortion”. They were told that this survey would inquire about their opinions on abortion in the United States. The survey was administered via the SurveyMonkey platform and random assignment was generated automatically through the platform’s computer logic. 
The survey consisted of five distinct components. After participants saw the informed consent and agreed to participate in the research, they completed a series of unrelated tasks. The first was a sentence reorganization activity. \citep{bakhti_religious_2018} Participants were randomly assigned to the religious prime or neutral condition. They were shown a group of five words and were instructed to construct a logical and coherent sentence using these words but could omit up to one of the words shown. In the religious prime condition, the stimuli included subtle mentions to religion through words such as “idol”, “sacred”, “pray”, “faith” and “miracle”. These words were absent in the neutral prime condition. The stimuli and answers are presented in Appendix A.

Next, participants are randomly assigned to read a statement on the US government’s role in abortion from a liberal or a conservative standpoint \citep{suhay_role_2018}. Arguments were presented to support the condition’s stance as shown in Appendix B. Upon finishing the short reading, participants were presented with a reading check to see if they understood the main idea of the reading. They were asked to rate the strength of the argument in their opinion, followed by the extent to which they agreed with it. These questions were measured on a 1 to 7 scale with 1 being strongly disagree and 7 being strongly agree. Participants were offered the opportunity to discuss their feelings about the issue and the reason they presented the ratings as they did. 

In the third section, participants were presented with a series of items regarding their religiosity \citep{bakhti_religious_2018}. They are presented a series of items regarding their feelings towards God and the importance of religion in their lives. As presented in Appendix C, each item is ranked from a 1 = strongly disagree to 7 = strongly agree scale. 

Participants were then asked a few demographic questions which addressed their political party affiliation, religious affiliation, frequency of participation in religious activities, gender identity, age, education, income, race and ethnicity. The final section consisted of a manipulation check to see if participants could figure out the purpose of the research. No one was able to state the research question nor purpose of research accurately for this component of the survey. Once the participants completed the survey, they were thanked, debriefed and paid. 

\subsection{Measures}

\subsubsection{Dependent Variable}

The dependent variable for this study is whether participants will lean conservative on their rantings of political issues, which was depicted by abortion in this study. A conservative measure would depend on the issue stance they were presented. Therefore, if the participant rated their agreement to a conservative stance at a rating for 5 or greater, this would be considered a conservative agreement. If participants rated a liberal stance at a rating of 3 or lower, this would also be considered conservative.

\subsubsection{Independent Variable}

The test variable of interest in this research is the presence of a subliminal religious prime in the first part of the survey. Roughly half of the participants were introduced to religious words such as “faith” and “pray” as they thought about how they would reorganize the sentences. This serves as a subliminal prime to religion as presented in political campaign advertisements. The prime is not as obvious tool to remind one of religiosity as voting in a church would, but the presence of this variable is to test the implications that such primes would have for political behavior.

A covariation variable is the argument that the participants are presented. Each participant was randomly assigned to see a conservative or liberal side of the debate regarding abortion. After reading the vignette, participants had to rate their agreement to the issue. The presence of these two sides on the abortion issues is necessary for our examination of the interaction between religious primes and presentation of conservative ideas to see if both factors to play a role in influencing conservative decisions.

\section{Results}

\subsection{Analysis on Religious Priming and Decision Results
}

For this study, a 2 $\times$ 2 ANOVA was conducted ($\alpha = 0.05$) to test the effects of the independent variables of religious prime and argument stance against individual opinions on the issue. We hypothesized that subliminal religious primes would influence more conservative voting. Through this test, we have substantial evidence to suggest that religious priming does influence conservative voting (F (3. 292) = 18.24, p $<$ .001, $\eta^{2}$ = 0.157793). However, given that this maps the general interaction of the conditions present in the survey, we looked deeper into the factors of the survey and found evidence that appeared to counter these results. While looking at the breakdown of the variables, the Tukey HSD post hoc test suggests that the argument that the participants saw had a greater impact on the difference that is seen in the overall interaction between the variables. We can see this if we run separate the data by argument and conduct a one-way ANOVA from people who only viewed the conservative condition (n = 150) versus those who view the liberal condition (n = 146). For the conservative condition, the results that there was no statistical significance between the conditions. (F (1,143) = 0.1014, p = 0.7506) This is also the case for the liberal condition (F (1,149) = 0.1963, p = 0.6583). The lack of statistical significance within each of the one-way ANOVAs suggests that subtle religious primes may not be significant enough to remind people of their religion. Factors that may explain for the presence of these observations arise from the demographics present among the people who participated in this research. Figure 1 shows the aggregate data for the number of people who rated each question to the questions on Appendix C. As we can see, 123 participants disagree with the fact that God exists. Participants overwhelmingly do not see religion as a main factor in their lives. In the open-ended religious affiliation question, many labeled themselves as not affiliating to a religion.  With these conditions, they are less likely to be susceptible to religious primes. 

With all the data that was gathered through the survey, the only factor that could be used to explain the significance observed in the overall interaction between the priming conditions and argument stance rests in the argument stance. Most of the participants in this study identify as a Democrat (n = 135). Therefore, they are more likely to side with the pro-life stance given their political beliefs once this schema is activated. This prompts them to answer based on these preferences, especially when the religious primes are not as strong and salient. The words used in this prime were “faith”, “pray”, and “miracle”, which may not trigger direct thoughts about God or the Bible. More importantly, as 183 participants disagree with the statement that religion influences their life choices, thus making it is very unlikely that reminders to a religious faith would change this predisposition. Therefore, while the argument stance led to significant results, we do not have sufficient evidence to support the accuracy of our hypothesis. In this case, subliminal religious primes do not appear to influence conservative decisions given the conditions that were explored in this section.

\subsection{Analysis on Abortion Rationales and Results
}

The study consisted of one open-ended question that asked participants to justify their stance on the issue. As I discussed before, the argument stance was the significant part in influencing individual support for the abortion case. People were not as influenced by the religious prime. Through this open-ended question, the responses that the participants generated can give us some hints as we understand possibilities regarding why the religious primes were not as impactful. 
For the participants who were exposed to the liberal argument (Appendix B), they saw arguments as to why abortions should be legal. For the open-ended question, the majority of the participants responded in agreement to this issue. They were more likely to express favor towards the right for people to choose what happens to a woman’s body. Many of the answers cited personal beliefs and referenced a woman’s right to choose. Of the people who saw this argument stance, many (n = 77) support the Democratic party. Therefore, the political rhetoric that they were accustomed to makes the argument towards being pro-choice. Therefore, the arguments presented in this survey would not do much to sway them, and this is evident in the responses. However, there were a handful of Republicans and Independents who saw this argument stance. Their responses paralleled their personal dispositions and religious beliefs. One participant cited their belief in God as their reason for opposing the arguments made in the passage. As more people used their personal beliefs to answer this question, it speaks to the influence of the argument and their daily exposure to the issue as the guiding force in making these decisions. 

On the other hand, another group of participants were exposed to the conservative side of the debate that claims that abortion should be illegal. In this group of responses, it is evident that some were influenced by the strength of the argument, and this is reflected in their answers. For those who disagree with the statement, they were more likely to provide a longer explanation to show their distaste, which contrasts those who disagreed with the legal stance as discussed in the previous paragraph. One participant said “I disagree with the statement because I support a woman's right to choose. It is not right to tell a woman how make medical decisions for herself, or to bring a child into the world that is not ready to be supported. Unwanted children do not get the quality of life they deserve and are a huge burden on families and society. Abortions are not a GOOD thing but are often better than the alternative and women should be able to make that decision for themselves.” From this response and many similar others, a pattern of cognitive dissonance was observed and is more salient than those who were against the liberal argument. Additionally, the presence of the conservative argument made people be more open to weighing out the costs and benefits of abortions, which was a pattern not present in the liberal section. In the responses that followed the liberal stance, participants were more likely to say, “I agree that abortion should be legal”, but in the conservative section, participants who saw this argument were more likely to respond with a rationale for their support of this argument or a denial. 

On both conditions of the argument exposure, people who were more likely to provide rationales were those who were generally pro-life. And from these responses, there were more mentions to religion and the idea that a baby is in existence because God mandated for them to be there. However, there were few of these arguments compared to the more secular arguments that most provided. Therefore, the presence of these answers suggests that personal predispositions were more helpful in influencing the decision, especially on an issue as sensitive as abortion. As people become more religious, their dispositions govern their judgments on everyday issues. However, as the political atmosphere moves in a more polarized direction \citep{abramowitz2010disappearing, mccarty2016polarized}, many other factors serve to govern decisions of individual actors in politics. The responses to this question serve as one such example of this phenomena that can be significant for future research. 

\subsection{Analysis of the Influence of Religious Primes on Religious Elites
}

The null results in the previous analysis may have resulted from the characteristics of individuals who utilize the Amazon Mechanical Turk platform. As discussed in the participant characteristics section, many of the respondents tend to be younger, Democrat, and better educated. This phenomena may work to hide the intentions of religious Americans represented in this survey. Therefore, in a separate 2 $\times$ 2 ANOVA under the same conditions of the original data analysis, I wanted to understand whether self-identified religiosity influences conservative voting when such identity is primed. I determined religious individuals to be those who answered 5 or higher on the 1 (strongly disagree) to 7 (strongly agree) scale for the item “I consider myself a religious person”. From the analysis, there is no significance (F (3,84) = 1.164, p = 0.3285, $\eta^{2}$ = 0.04086) in the interaction between the condition, the argument, and the extent to which the participant agreed with the argument. Therefore, religious primes in this case, did not influence individuals, even those who were the most religious, to move more to the right on the political spectrum from their current position. There are extraneous factors, including individual political ideology, that may have influenced the decisions that the participants made in the simulation. 

\section{General Discussion and Conclusion
}

From the discussion of literature to the results, my findings do not parallel previous research results of the influence of religion in political decision-making \citep{calfano_god_2009, rutchick_deus_2010}. In this study, I do not have significant results that support my hypothesis. While there is significance in the interaction between argument stance and religious primes, individual predispositions are the greatest lurking variables that explained their ratings on the dependent score \citep{mclaughlin_cueing_2014}. People who were primed with subtle reminders to religion were more prone to more conservative stances and these primes did not influence overall voting decisions. I entered this study with a question regarding the ways that religious primes would influence conservative voting but left with more questions that can pave way for future research. 

Through the present research, we see that personal dispositions, while not made salient through primes, governed individual decision making. The sample employed in this research were predominately agnostic or atheist with a Democratic background. Most of the participants had some college or more in terms of education, making them more liberal minded than the base of Republican party \citep{gelman_red_2010}. This representation of the general population introduces a limitation in the research. As participants were self-selected and gave their consent for participation, it opens the possibility that the data would not be a grasp of the US population. Furthermore, people who would work for Amazon Mechanical Turk often have skills and knowledge that is not evenly spread across the country due to limits to access in information and technology that are needed for the job \citep{hochschild2018strangers}. For future research, it would be useful to sample a more representative group of voters through questions posted to national election surveys to gain a holistic view on the direct influence of location primes on polling behavior. By understanding where people actually voted and how they voted in a nationwide representative survey can shine more light into the phenomena. In doing so, we are not blocking out people without the skill sets to use MTurk, but we may open the results to more influences by lurking variables. 

The open-ended responses suggest that personal dispositions are more likely to shape decisions versus the subtle prime to religion that was presented in this study. When people rated the extent that they agreed with the arguments for or against abortion, many used their predispositions. As I discussed in the results section, those who saw the conservative stance were likely to reflect cognitive dissonance through their need to reason out why their stance if it is contrary to the one presented, is more correct. From these answers, we see that religion only works for those who are religious. For those few participants, the salience of religion governing their life decisions is shown in their answers through their references to God and morality. This gives us further evidence that personal dispositions and argument strength were guiding factors in influencing the dependent variable, which creates a limitation to the design of this study.

Another limitation is the salience of the prime. Voting in a church can serve as a constant reminder to religiosity as people think about the name of the place as they search for it down the road. Going to a church to vote may remind people of their commutes to their weekly Sunday sermons. These personal connections can trigger reminders of their friends at church and memories of their conversations. As the content of religiosity becomes more salient, it would be a more impactful reminder towards this part of their individual identity than rearranging words that mildly mention religion through ideas of faiths and miracles. These vague terms employed through the task may not be enough to trigger such memories of teachings or congregation social networking. A path for future research would be to make these primes more salient. By asking participants to give directions to a local church from their house, or drive on the road to church via a simulator, it can lead to more vivid memories of the teachings that research describes as the greatest indicator to religious voting \citep{putnam_american_2010}.

Due to resource limitations, Amazon Mechanical Turk served as a relatively cheap platform to allow the researcher to simulate real-world phenomena. The limitation that this presents is that there is great variation to the situation in which people are taking these surveys, which is a limitation that I had little control over. As the main purpose of this study is to see if the location, or primes of a location, would influence voting, variations of the setting can lead to differences in results. If people completed the survey in a church, it may make the prime more salient. However, if they complete this in a noisy café with their friends, it may lead them to give lower quality results or encourage them to click random buttons such that they would be paid the award. In future research, it would be helpful to control the setting even if an online database was used to collect the data. By controlling the location, it would reduce the number of confounding variables that influenced the results of the study. 

Overall, while there was no significance relating the primes of religion to the resulting conservative vote, we see significance in the farming of the argument in influencing political behavior. This goes to show that individual dispositions towards the issues were stronger as the primes to religion may not be salient enough. Additionally, people may be influenced by the environment in which this survey was actually taken. For future research, it would be helpful to control the setting should an online survey program be used. Or, another possibility would be to collect exit poll data outside a church and compare it to exit poll data of voters who vote in a library or school. Despite its limitations and lack of significance on the main variable, this study does go to show that the framing of an issue can influence decisions and personal factors play a strong role in shaping political opinion. 

\bibliography{Priming}

\end{document}